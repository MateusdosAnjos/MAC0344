\documentclass[12pt]{article}

\usepackage[utf8]{inputenc}
\usepackage{graphicx}
\usepackage[portuguese]{babel}
\usepackage{float}

\title{MAC0344 Arquitetura de Computadores\\
Lista de Exercícios No. 3
}
\author{Mateus Agostinho dos Anjos\\NUSP 9298191}
\date{\today}

\begin{document}
	\maketitle
	\begin{itemize}
		\item[\textbf{1 -}]
			\hfill\newline
			Temos as seguintes instruções:		
			\begin{center}
				\begin{tabular}{cl}
					1: & $A = B + 2$\\
					2: & $B = C$\\
					3: & $E = A + B$\\
					4: & $B = 0$\\
					5: & $F = B + 1$\\				
				\end{tabular}			
			\end{center}
			Identificamos as dependências utilizando:\\
			\begin{center}
				\begin{tabular}{ll}
			$x \ \longrightarrow^v \ y$ & Dependência verdadeira\\
			$x \ \longrightarrow^{anti} \ y$ & Anti-dependência\\
			$x \ \longrightarrow^{saida} \ y$ & Dependência de saída\\		
				\end{tabular}			
			\end{center}
			Onde $x$ e $y$ são as instruções {1, 2, 3, 4, 5} listadas
			acima.\\
			
			Portanto temos:\\
			\begin{center}
				$1 \ \longrightarrow^v \ 3$\\
				$2 \ \longrightarrow^v \ 3$\\
				$4 \ \longrightarrow^v \ 5$\\
				\hfill\newline
				$1 \ \longrightarrow^{anti} \ 2$\\
				$1 \ \longrightarrow^{anti} \ 4$\\
				$3 \ \longrightarrow^{anti} \ 4$\\
				\hfill\newline
				$2 \ \longrightarrow^{saida} \ 4$\\
			\end{center}
		\item[\textbf{2 -}]
			\hfill\newline
			Para remover as anti-dependências e as dependências de saída 
			devemos fazer o seguinte:
			\begin{center}
				\begin{tabular}{cl}
					1: & $A = B + 2$\\
					2: & $E = A + C$\\
					3: & $B = 0$\\
					4: & $F = B + 1$\\				
				\end{tabular}			
			\end{center}
	\end{itemize}
\end{document}