\documentclass[12pt]{article}

\usepackage[utf8]{inputenc}
\usepackage{graphicx}
\usepackage[portuguese]{babel}
\usepackage{float}

\title{MAC0344 Arquitetura de Computadores\\
Lista de Exercícios No. 4
}
\author{Mateus Agostinho dos Anjos\\NUSP 9298191}
\date{\today}

\begin{document}
	\maketitle
	\begin{itemize}
		\item[\textbf{1 -}]
			\hfill\newline
			Começamos o código de Hamming definindo os valores de $x_1$ até $x_{11}$.
			\begin{center}
				\begin{tabular}{ccccc}
					$x_1$ & = & a determinar & = & ?\\
					$x_2$ & = & a determinar& = & ?\\
					$x_3$ & = & $m_1$ & = & 1 \\
					$x_4$ & = & a determinar & = & ?\\
					$x_5$ & = & $m_2$ & = & 1\\
					$x_6$ & = & $m_3$ & = & 0\\
					$x_7$ & = & $m_4$ & = & 0\\
					$x_8$ & = & a determinar & = & ?\\
					$x_9$ & = & $m_5$ & = & 1\\
					$x_{10}$ & = & $m_6$ & = & 0\\
					$x_{11}$ & = & $m_7$ & =& 1\\
				\end{tabular}
			\end{center}
			Agora calculamos $x_1, \ x_2, \ x_3, \ x_4$ da seguinte forma:\\
			($\oplus$ representa a operação "ou exclusivo" (XOR))
			\begin{center}
				\begin{tabular}{ccl}
					$x_1$ & = & $x_3 \oplus x_5 \oplus x_7 \oplus x_9 \oplus x_{11}$\\
					$x_2$ & = & $x_3 \oplus x_6 \oplus x_7 \oplus x_{10} \oplus x_{11}$\\
					$x_4$ & = & $x_5 \oplus x_6 \oplus x_7 \oplus x_{12}$\\
					$x_8$ & = & $x_9 \oplus x_{10} \oplus x_{11} \oplus x_{12}$\\
				\end{tabular}
			\end{center}
	\end{itemize}
\end{document}