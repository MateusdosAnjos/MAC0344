\documentclass[12pt]{article}

\usepackage[utf8]{inputenc}
\usepackage{graphicx}
\usepackage[portuguese]{babel}
\usepackage{float}
\usepackage{xcolor}

\title{MAC0344 Arquitetura de Computadores\\
Lista de Exercícios No. 4
}
\author{Mateus Agostinho dos Anjos\\NUSP 9298191}
\date{\today}

\begin{document}
	\maketitle
	\begin{itemize}
		\item[\textbf{1 -}]
			\hfill\newline
			Começamos o código de Hamming definindo os valores de $x_1$ até $x_{11}$.
			\begin{center}
				\begin{tabular}{ccccc}
					$x_1$ & = & a determinar & = & ?\\
					$x_2$ & = & a determinar& = & ?\\
					$x_3$ & = & $m_1$ & = & 1 \\
					$x_4$ & = & a determinar & = & ?\\
					$x_5$ & = & $m_2$ & = & 1\\
					$x_6$ & = & $m_3$ & = & 0\\
					$x_7$ & = & $m_4$ & = & 0\\
					$x_8$ & = & a determinar & = & ?\\
					$x_9$ & = & $m_5$ & = & 1\\
					$x_{10}$ & = & $m_6$ & = & 0\\
					$x_{11}$ & = & $m_7$ & =& 1\\
				\end{tabular}
			\end{center}
			Agora calculamos $x_1, \ x_2, \ x_3, \ x_4$ da seguinte forma:\\
			($\oplus$ representa a operação "ou exclusivo" (XOR))
			\begin{center}
				\begin{tabular}{ccl}
					$x_1$ & = & $x_3 \oplus x_5 \oplus x_7 \oplus x_9 \oplus x_{11}$\\
					$x_2$ & = & $x_3 \oplus x_6 \oplus x_7 \oplus x_{10} \oplus x_{11}$\\
					$x_4$ & = & $x_5 \oplus x_6 \oplus x_7$\\
					$x_8$ & = & $x_9 \oplus x_{10} \oplus x_{11}$\\
				\end{tabular}
			\end{center}
			\newpage
			Existe uma forma simples para chegar às fórmulas, basta seguir os passos:\\
			1. escrever os números de 1 a 11 em binário.\\
			2. $x_1$ é calculado utilizando os números que possuem o bit $2^0$ igual a 1.\\
			3. $x_2$ é calculado utilizando os números que possuem o bit $2^1$ igual a 1.\\
			4. $x_3$ é calculado utilizando os números que possuem o bit $2^2$ igual a 1.\\
			5. $x_4$ é calculado utilizando os números que possuem o bit $2^3$ igual a 1.\\
			
			Depois do cálculo da fórmula acima, chegamos em:
			\begin{center}
				\begin{tabular}{ccl}
					$x_1$ & = & 0\\
					$x_2$ & = & 1\\
					$x_4$ & = & 1\\
					$x_8$ & = & 0\\
				\end{tabular}
			\end{center}
			Portanto o código de Hamming $x_1x_2x_3x_4x_5x_6x_7x_8x_9x_{10}x_{11}$ para 
			o dado $m_1m_2m_3m_4m_5m_6m_7 = 1100101$ será:
			\begin{center}
				\begin{tabular}{ccc}
					$x_1$ & = & 0\\
					$x_2$ & = & 1\\
					$x_3$ & = & 1 \\
					$x_4$ & = & 1\\
					$x_5$ & = & 1\\
					$x_6$ & = & 0\\
					$x_7$ & = & 0\\
					$x_8$ & = & 0\\
					$x_9$ & = & 1\\
					$x_{10}$ & = & 0\\
					$x_{11}$ & = & 1\\
				\end{tabular}
			\end{center}
		\newpage
		\item[\textbf{2 -}]
		\hfill\newline
			Para detectar erros primeiro devemos comparar cada $x_\alpha$ com seu respectivo $y_\alpha$,
			veja a tabela de comparação abaixo:
			\begin{center}
				\begin{tabular}{ccc|ccc}
					$y_1$ & = & 0 & 0 & = & $x_1$\\
					\color{red}$y_2$ & \color{red} = & \color{red}0 &
					\color{red}1 & \color{red} = & \color{red}$x_2$\\
					$y_3$ & = & 1 & 1 & = & $x_3$\\
					$y_4$ & = & 1 & 1 & = & $x_4$\\
					\color{red}$y_5$ & \color{red} = & \color{red} 0 &
					\color{red} 1 & \color{red} = & \color{red} $x_5$\\
					$y_6$ & = & 0 & 0 & = & $x_6$\\
					$y_7$ & = & 0 & 0 & = & $x_7$\\
					$y_8$ & = & 0 & 0 & = & $x_8$\\
					$y_9$ & = & 1 & 1 & = & $x_9$\\
					$y_{10}$ & = & 0 & 0 & = & $x_{10}$\\
					$y_{11}$ & = & 1 & 1 & = & $x_{11}$\\
				\end{tabular}
			\end{center}
			Em \color{red}vermelho \color{black} vemos as linhas em que $y_\alpha$ é diferente
			de $x_\alpha$.\\
			Como existe $\alpha$ tal que $x_\alpha \neq y_\alpha$, então \textbf{há um erro}.
	\end{itemize}
\end{document}