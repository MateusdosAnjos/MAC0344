\documentclass[12pt]{article}

\usepackage[utf8]{inputenc}
\usepackage{graphicx}
\usepackage[portuguese]{babel}
\usepackage{float}
\usepackage{xcolor}

\title{MAC0344 Arquitetura de Computadores\\
Lista de Exercícios No. 5
}
\author{Mateus Agostinho dos Anjos\\NUSP 9298191}
\date{\today}

\begin{document}
	\maketitle
    A instrução NOVA faz:
    \begin{itemize}
        \item[-]
            Lê a memória M[0] e coloca o valor lido em AC,
        \item[-]
            Multiplica SP por 16 e coloca o resultado em SP,
        \item[-]  
            Calcula o valor SP como sendo SP + AC,
        \item[-] 
            Se SP for zero então soma 1 a AC
            caso contrário faz AC igual a zero.
        \item[-]
            Retorna à posição 0.
    \end{itemize}

    Escrevendo em micro-assembler o trecho das micro-instruções que 
    correspondem a execução de NOVA e supondo que o início desse trecho é na 
    linha 21, temos:

    \begin{itemize}
        \item[-]
            21: mar := 0; rd\\
            22: rd\\
            23: ac := mbr\\
        \item[-]
            24: sp := lshift(sp + sp) \# Multiplica por 4\\
            25: sp := lshift(sp + sp) \# Multiplica por 4\\
        \item[-]  
            26: sp := sp + ac; if z then goto ??\\
        \item[-]
            27: ac := ac + 1\\
            28: ac := 0\\
        \item[-]
            29: goto 0\\
    \end{itemize}



\end{document}