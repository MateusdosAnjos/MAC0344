\documentclass[12pt]{article}

\usepackage[utf8]{inputenc}
\usepackage{graphicx}
\usepackage[portuguese]{babel}
\usepackage{float}
\usepackage{xcolor}

\title{MAC0344 Arquitetura de Computadores\\
Lista de Exercícios No. 5
}
\author{Mateus Agostinho dos Anjos\\NUSP 9298191}
\date{\today}

\begin{document}
	\maketitle
    A instrução NOVA faz:
    \begin{itemize}
        \item[-]
            Lê a memória M[0] e coloca o valor lido em AC,
        \item[-]
            Multiplica SP por 16 e coloca o resultado em SP,
        \item[-]  
            Calcula o valor SP como sendo SP + AC,
        \item[-] 
            Se SP for zero então soma 1 a AC
            caso contrário faz AC igual a zero.
        \item[-]
            Retorna à posição 0.
    \end{itemize}

    Escrevendo em micro-assembler o trecho das micro-instruções que 
    correspondem a execução de NOVA e supondo que o início desse trecho é na 
    linha 21, temos:

    \begin{center}
        \begin{tabular}{ll}
            21: &   mar := 0; rd                                \\
            22: &   sp := lshift(sp + sp); rd                   \\
            23: &   ac := mbr                                   \\
            24: &   sp := lshift(sp + sp)                       \\
            25: &   sp := sp + ac; if z then goto 27            \\
            26: &   ac := 0; goto 0                             \\
            27: &   ac := ac + 1; goto 0                        \\
        \end{tabular}
    \end{center}
    
\end{document}